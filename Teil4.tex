\section{Teil}
beinhaltet folgende Foliensätze:

\begin{itemize}
    \item Teil 4:  Systemverhalten (Grundlagen der Modellplanung und –Bildung und Simulation)

\end{itemize}

% subsection
%-------------------------------------------------------------------------------------------
\subsection{Was ist ein Modell?}
Modelle sind Abstraktionen und Vereinfachung der Realität und zeigen deshalb nur Teilaspekte auf.
Es ist daher wichtig, dass die Modelle im Hinblick auf die Situation und die Problemstellung aussagekräftig sind.
Dies bedeutet, dass bei allen Überlegungen die Frage nach der Zweckmäßigkeit und der Problemrelevanz zu stellen ist.
\begin{figure}[H]
    \centering
    \includegraphics[width=0.6\linewidth]{Bilder/Teil4_ModelleBeispiel1.png}
    \caption{Beispiel Modelle verschiedener Lebensbereiche}
\end{figure}
% subsection
%-------------------------------------------------------------------------------------------
\subsection{Was bedeutet Problemlösung durch Abstraktion (Modellbildung) und Interpretation?}
Vorgehensweise der Modellbildung wird auf eine abstrakte Ebene verlagert, um die Lösungsfindung zu vereinfachen. 

\textbf{Zielgerichtete Vereinfachung durch Abstraktion.}

\begin{figure}[H]
    \centering
    \includegraphics[width=0.6\linewidth]{Bilder/Teil4_ProblemlösungdurchAbstraktionundInerpretation.png}
    \caption{Problemlösung durch Abstraktion und Interpretation}
\end{figure}

% subsection
%-------------------------------------------------------------------------------------------
\newpage
\subsection{Welche Verfahren der Modellbildung gibt es?}
\begin{itemize}
    \item \textbf{Rechnerische Verfahren:}\\
    Dazu werden \textbf{mathematische Modelle} benötigt, die formal durch \textbf{Gleichungen} beschrieben werden.
    
    Heutzutage stehen neben den traditionellen Methoden numerische und symbolische Softwareprogramme zu Verfügung.
    
    \item \textbf{Experimentelle bzw. messtechnische Verfahren:}\\
    Zu ihrer Anwendung werden \textbf{physikalische (experimentelle) Modelle} benötigt, an denen Versuche, Messungen und Auswertungen
    durchgeführt werden können. Dadurch sollen wesentliche Einflüsse messtechnisch erfasst werden.
    
    Im Maschinenbau etwa sind dies typischerweise Prototypen, Testobjekte, Versuchsanordnungen oder
    maßstäbliche Modelle.

    \item \textbf{Hybride Verfahren} (Kombination von Berechnung und Experiment bzw. Messung):\\
    Diese Verfahren \textbf{nutzen sowohl Messgrößen as auch mathematische Modelle.} Während bei rechnerischen Verfahren Fehler aufgrund
    von Modellierungsungenauigkeiten auftreten, sind bei experimentellen Verfahren mehr oder weniger große Messfehler unvermeidbar.
    
    Liegen sowohl mathematische Modelle als auch Messergebnisse vor, so kann man versuchen, Hypothesen über die Art der Fehler 
    zu bilden und das Modell anhand der Messergebnisse zu verbessern.
\end{itemize}

% subsection
%-------------------------------------------------------------------------------------------
\subsection{Wie ist der Ablauf zur Entstehung eines rechnerinternen Modells?}
\begin{itemize}
    \item Der Modellbildner entwickelt eine gedankliche Vorstellung des zu untersuchenden Originals (z.B. reales
    objekt oder neues Produkt) in Form eines mentalen Modells (Gedankenmodells), das anschließend 
    zu seiner Erfassung in eine formalisierte Informationsform mit Hilfe von Informationselementen und -strukturen gebracht wird.
    \item Dieses \glqq Informationsmodell\grqq wird am Rechner implementiert (rechnerinternes Modell).
\end{itemize}
\begin{figure}[H]
    \centering
    \includegraphics[width=0.3\linewidth]{Bilder/Teil4_RechnerinternesModell.png}
    \caption{Ablauf zur Entstehung eines rechnerinternen Modells}
\end{figure}

% subsection
%-------------------------------------------------------------------------------------------
\subsection{Was ist bei der Interpretation von Ergebnissen wichtig?}
Eine wichtige Aufgabe des Produktentwicklers ist die 
Interpretation der Ergebnisse. Dazu muss er zwischen physikalischen Phänomenen und künstlichen Effekten 
(Artefakten), die z.B. von Messfehlern bzw. numerischen Lösungsverfahren herrühren, unterscheiden können. 

Dies erfordert Kenntnisse und Erfahrung sowohl über die untersuchten Fragestellungen als auch über die Verfahren, die zur Lösung des Modellproblems verwendet werden. 

Einfache, überschlägige Abschätzungen und Erfahrung sind unerlässlich, um die Ergebnisse (Messergebnisse bzw. Rechenergebnisse) 
zu überprüfen sowie zu bewerten und damit hohe Qualität der Simulation sicherzustellen.

% subsection
%-------------------------------------------------------------------------------------------
\subsection{Was bedeutet Simulation und wofür ist dies relevant?}

Simulation ist ein Verfahren zur Nachbildung eines Systems mit seinen 
dynamischen Prozessen in einem experimentierbaren Modell, um zu 
Erkenntnissen zu gelangen, die auf die Wirklichkeit übertragbar sind. 

Im weiteren Sinne wird unter Simulation das Vorbereiten, Durchführen und 
Auswerten gezielter Experimente mit einem Simulationsmodell verstanden. 

Mit Hilfe der Simulation kann das zeitliche Ablaufverhalten komplexer Systeme 
untersucht werden.    

Simulationen werden durchgeführt wenn:
\begin{itemize}
    \item kein reales System verfügbar ist (Entwurfsphase)
    \item das Experiment am realen System zu lange dauert
    \item das Experiment am realen System zu teuer ist
    \item das Experiment am realen System zu gefährlich ist (Flugzeug, Kraftwerk)
    \item die Zeitkonstanten des realen Systems zu groß sind (Klimamodelle)
\end{itemize}